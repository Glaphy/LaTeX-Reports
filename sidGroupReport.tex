\documentclass[a4paper]{article}
\usepackage{amsmath}
\usepackage{amssymb}
\usepackage{bm}
\usepackage{graphicx}

\newcommand{\mat}[1]{\mathbf{#1}}
\renewcommand{\vec}[1]{\bm{#1}}

\begin{document}
%\title{Sid's Group Report}
%\author{Siddhartha Menon}
%\date{\today}
%
%\maketitle

\section{Introduction to the Direct Method}
The direct method is possibly the most logically straightforward way to solve
for the values of the mesh points. It involves expressing the relationships of
the various grid points with each other through a system of simultaneous
equations, and then solving this system to obtain a vector containing the
solutions to the mesh points. Thus, in contrast to the iterative methods,
results are obtained at their final accuracy rather than being successively
improved over several cycles.

In our implementation we consider as system of the form:
\begin{align*}
	-4&V_1+V_2+0V_3+0V_4+V_5+\dots+0V_{16}&=0\\
	&V_1-4V_2+V_3+0V_4+0V_5+V_6+\dots+0V_{16}&=0\\
	&\vdots\\
	0&V_1+\dots+V_{12}+0V_{13}+0V_{14}+V_{15}-4V_{16}&=0\\
\end{align*}
Which can be expressed by the matrix equation:
\begin{equation*}
	\mat{A}\vec{v}=\vec{b}
\end{equation*}
Where $\mat{A}$ is a matrix holding the coefficients of the system, $\vec{v}$
holds the voltage solutions, and $\vec{b}$ represents the right hand values of
the system. We could now apply a variety of generic matrix solving techniques
to find $\vec{v}$ but these methods quickly become impossible to practically
implement in larger problems for reasons discussed in the following sections.

\section{Implementation of the Direct Method}
The first thing to notice is that if we have a $N\times N$ grid then the
coefficient matrix has $N^4$ elements. This quartic growth means that the
storage space required quickly overtakes any realistic memory capacity.
Furthermore, even if one could procure such an expansive memory device, we
would still leave much to be desired in terms of efficiency. To understand why
this is, consider the structure of the matrix $\mat{A}$. We can notice that on
any given line there are at most five non zero elements: the point of interest,
and the four points around it. The rest of the elements on that row are
initialised to zero. The net effect of this is that the matrix $\mat{A}$ is
\emph{extremely} sparse, especially for large $N$. In fact the sparseness of
the matrix is the principle factor on which optimisations are made. In the next
section we discuss methods of storing large sparse matrices and the challenges
they present.

\subsection{Storage Sytems for Sparse Matrices}






\end{document}
